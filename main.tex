%%%%%%%%%%%%%%%%%%%%%%%%%%%%%%%%%%%%%%%%%
% Masters/Doctoral Thesis 
% LaTeX Template
% Version 2.2 (21/11/15)
%
% This template has been downloaded from:
% http://www.LaTeXTemplates.com
%
% Version 2.x major modifications by:
% Vel (vel@latextemplates.com)
%
% This template is based on a template by:
% Steve Gunn (http://users.ecs.soton.ac.uk/srg/softwaretools/document/templates/)
% Sunil Patel (http://www.sunilpatel.co.uk/thesis-template/)
%
% Template license:
% CC BY-NC-SA 3.0 (http://creativecommons.org/licenses/by-nc-sa/3.0/)
%
%%%%%%%%%%%%%%%%%%%%%%%%%%%%%%%%%%%%%%%%%

%----------------------------------------------------------------------------------------
%	PACKAGES AND OTHER DOCUMENT CONFIGURATIONS
%----------------------------------------------------------------------------------------
\documentclass[
11pt, % The default document font size, options: 10pt, 11pt, 12pt
%oneside, % Two side (alternating margins) for binding by default, uncomment to switch to one side
spanish, % ngerman for German
singlespacing, % Single line spacing, alternatives: onehalfspacing or doublespacing
%draft, % Uncomment to enable draft mode (no pictures, no links, overfull hboxes indicated)
%nolistspacing, % If the document is onehalfspacing or doublespacing, uncomment this to set spacing in lists to single
%liststotoc, % Uncomment to add the list of figures/tables/etc to the table of contents
%toctotoc, % Uncomment to add the main table of contents to the table of contents
parskip, % Uncomment to add space between paragraphs
%nohyperref, % Uncomment to not load the hyperref package
headsepline, % Uncomment to get a line under the header
]{MastersDoctoralThesis} % The class file specifying the document structure

\usepackage[utf8]{inputenc} % Required for inputting international characters
\usepackage[T1]{fontenc} % Output font encoding for international characters

\usepackage{palatino} % Use the Palatino font by default

%\usepackage[backend=bibtex,style=authoryear,natbib=true]{biblatex} % User the bibtex backend with the authoryear citation style (which resembles APA)
%\addbibresource{example.bib} % The filename of the bibliography
\usepackage[numbers]{natbib}
\usepackage{hyperref}
\usepackage{listings}
\usepackage{array}

\hypersetup{pdfpagemode={UseOutlines},
	bookmarksopen=true,
	bookmarksopenlevel=0,
	hypertexnames=false,
	colorlinks=true, % Set to false to disable coloring links
	citecolor=black, % The color of citations
	linkcolor=black, % The color of references to document elements (sections, figures, etc)
	urlcolor=RoyalBlue, % The color of hyperlinks (URLs)
	pdfstartview={FitV},
	unicode,
	breaklinks=true,
}

\pdfstringdefDisableCommands{% If there is an explicit linebreak in a section heading (or anything printed to the pdf-bookmarks), it is replaced by a space
	\let\\\space%
}

\usepackage{xstring}
\usepackage{todonotes}
\usepackage{pdfpages}
\usepackage{epigraph}
\usepackage{framed}


\let\originalbibitem\bibitem
\def\bibitem#1#2\par{
	\noexpandarg
	\originalbibitem{#1}
	\StrSubstitute{#2}{C.-F. San-Nicolas-Martinez}{\textbf{C.-F. San-Nicolas-Martinez}}\par}
\usepackage[autostyle=true]{csquotes} % Required to generate language-dependent quotes in the bibliography

\addto\captionsspanish{
	\renewcommand{\tablename}{Tabla}
	\renewcommand{\listtablename}{Índice de tablas}
	\renewcommand{\appendixname}{Anexo}
}


%----------------------------------------------------------------------------------------
%	MARGIN SETTINGS
%----------------------------------------------------------------------------------------

\geometry{
	paper=a4paper, % Change to letterpaper for US letter
	inner=3cm, % Inner margin
	outer=2.5cm, % Outer margin
	bindingoffset=0.5cm, % Binding offset
	top=2.5cm, % Top margin
	bottom=2.5cm, % Bottom margin
	%showframe,% show how the type block is set on the page
}
\setlength{\parindent}{8mm}
\setcounter{secnumdepth}{3} % para que ponga 1.1.1.1 en subsubsecciones
\setcounter{tocdepth}{3} % para que ponga subsubsecciones en el indice

%----------------------------------------------------------------------------------------
%----ACRONYMS
%-------------------------------------------------------------------------------------

\begin{document}

\frontmatter % Use roman page numbering style (i, ii, iii, iv...) for the pre-content pages

\pagestyle{plain} % Default to the plain heading style until the thesis style is called for the body content

%----------------------------------------------------------------------------------------
%	TITLE PAGE
%----------------------------------------------------------------------------------------

\begin{titlepage}
	\begin{center}
		UNIVERSIDAD POLITÉCNICA DE CARTAGENA\\
		\vspace*{0.2in}
		\begin{figure}[ht!]
			\centering
			\includegraphics[width=0.5\linewidth]{imagenes/logoupct.jpg}
		\end{figure}
		ESCUELA TÉCNICA SUPERIOR DE INGENIERÍA DE TELECOMUNICACIÓN\\
		\vspace*{0.35in}
		\begin{large}
			Trabajo Fin de Máster\\
		\end{large}
		\vspace*{0.25in}
		\rule{80mm}{0.1mm}\\
		\vspace*{0.3in}
		\begin{Large}
			\textbf{Desarrollo de APIs para escenarios SDN-NFV} \\
		\end{Large}
		\vspace*{0.35in}
		\rule{80mm}{0.1mm}\\
		\vspace*{0.55in}
	\end{center}
    \vfill
	\begin{flushleft}
		\begin{large}
              \textit{\textbf{Autor:}} César Francisco San Nicolás Martínez\\
			  \textit{\textbf{Director:}} Pablo Pavón Mariño\\
			  \textit{\textbf{Codirector:}} Francisco Javier Moreno Muro\\
		\end{large}
	\end{flushleft}
	\vspace*{0.7in}
	\begin{center}
		FECHA\\
	\end{center}
    \vfill
	
\end{titlepage}
\restoregeometry
\cleardoublepage


\begin{center}
	\vspace*{3cm}
	\fbox{
	\begin{minipage}[][][s]{0.6\linewidth}
		\begin{tabbing}
			\hspace{6.5cm}\=\kill 
			\textbf{Autor:}\>César Francisco San Nicolás Martínez\\ 
			\textbf{E-mail del autor:}\>cesarfsannicolasmartinez@gmail.com\\
			\textbf{Director:} \>Pablo Pavón Mariño \\
			\textbf{E-mail del director:} \> pablo.pavon@upct.es \\
			\textbf{Codirector:} \>Francisco Javier Moreno Muro \\
			\textbf{E-mail de codirector:} \> javier.moreno@upct.es \\
			\textbf{Título del TFM:} \> Desarrollo de APIs para escenarios SDN-NFV\\
		\end{tabbing} 
		\textbf{Resumen:}\\
		El paradigma SDN se ha convertido en la manera más eficiente de gestionar las redes actuales, gracias a la automatización de las funciones de operación y gestión. A su vez, el paradigma NFV se ha convertido en un recurso valioso para optimizar recursos de una infraestructura IT. Debido a esto, han ido surgiendo herramientas para la aplicación de las técnicas SDN y NFV. El objetivo de este proyecto es desarrollar diversos clientes para controlar remotamente las herramientas ONOS, OSM (Open Source MANO) y OpenStack e integrarlos en un plugin de la herramienta Net2Plan. \\

		\begin{tabbing}
			\hspace{6.5cm}\=\kill 
			\textbf{Titulación:}\>Máster en Ingeniería de Telecomunicación\\
			\textbf{Departamento:}\> Tecnologías de la Información y las Comunicaciones \\
			\textbf{Fecha de presentación:}\> FECHA\\
		\end{tabbing} 
	\end{minipage}%
}
\end{center}

\tableofcontents % Prints the main table of contents
%\listoffigures % Print the list of figures


\mainmatter % Begin numeric (1,2,3...) page numbering

\pagestyle{thesis} % Return the page headers back to the "thesis" style

% Include the chapters of the thesis as separate files from the Chapters folder
% Uncomment the lines as you write the chapters
\chapter{Introducción}

El paradigma \ac{SDN} se ha convertido en una de las maneras más eficientes de gestionar las redes de telecomunicación, permitiendo a los administradores de red una gestión y configuración a alto nivel. Todo esto es gracias a la naturaleza de \ac{SDN}, cuya principal premisa es desacoplar totalmente el plano de control del plano de datos.

El paradigma \ac{NFV} se ha convertido en una de las tecnologías más valiosas para optimizar recursos de una infraestructura \ac{ICT}. Al igual que \ac{SDN}, \ac{NFV} permite a los administradores de red gestionar los diferentes recursos a un alto nivel. Gracias a la naturaleza de NFV, que se basa en la idea de virtualizar cualquier función de red, como puede ser un \textit{firewall} o un dispositivo \ac{NAT}, sustituyendo un dispositivo físico por una máquina virtual que realice su misma función de una manera más eficiente.

Debido a esto, han ido surgiendo herramientas para la aplicación de las técnicas \ac{SDN} y \ac{NFV} para automatizar y optimizar las redes de telecomunicación.

\section{Motivaciones}

Este proyecto viene motivado por la evolución de las redes de telecomunicación en los últimos años. El concepto de red de telecomunicación se inició como algo puramente físico, constituido exclusivamente por dispositivos \textit{hardware}. Actualmente, una red de telecomunicación está compuesta por dispositivos físicos sustentados por aplicaciones \textit{software} que son utilizadas por los administradores de red para realizar tareas a alto nivel de forma automatizada.

Los paradigmas \ac{SDN} y \ac{NFV} han posibilitado esta evolución, permitiendo a los administradores de redes el poder configurar y gestionar una red a un gran alto nivel, así como el sustituir ciertos dispositivos físicos por máquinas virtuales que realicen su misma función dentro de un entorno de red, pero de manera más eficiente.

Para realizar todas estas tareas a alto nivel, son necesarias diferentes herramientas \textit{software} destinadas a tal fin. Cada una de estas herramientas tiene su propia interfaz de comunicación para poder hacer uso de ellas. Un problema que surge es la particularidad de las interfaces de comunicación de las distintas herramientas, ya que cada una de ellas tiene su propia sintáxis.

Así mismo, no existe una entidad central que permita gestionar las interacciones entre las diferentes herramientas de forma óptima. Por esto último surge la idea de desarrollar diferentes clientes \textbf{open-source} para interactuar con las distintas herramientas de una forma sencilla y transparente para el usuario, y convertir a Net2Plan en esa entidad central que se pueda comunicar con las diferentes herramientas y gestionar las interacciones entre ellas, para proporcionar optimización al sistema gracias a una visión más amplia de la red.

\section{Objetivos}

El objetivo principal de este trabajo es desarrollar diferentes \acp{API} \textbf{open-source} e integrarlas en un plugin de Net2Plan para llevar a cabo una prueba de concepto. Este objetivo se puede desglosar en otros más pequeños:

\begin{itemize}
	\item Adquisición de conocimientos de las distintas herramientas (\ac{ONOS}, Mininet, \ac{OSM} y OpenStack). 
	\item Desarrollo de \acp{API} para entornos \ac{SDN} (J-ONOS Client).
	\item Desarrollo de \acp{API} para entornos \ac{NFV} (J-OSM Client y J-OpenStack Client).
	\item Integración de todas las \acp{API} en un plugin de Net2Plan para una prueba de concepto.
	\item Obtención y análisis de resultados.
\end{itemize}

\section{Plan de trabajo}

Para la consecución de los objetivos marcados, el plan de trabajo del proyecto consta de diferentes fases, cada una de ellas destinada a cumplir un objetivo concreto:

\begin{itemize}
	\item Estudio y familiarización con \ac{ONOS}, Mininet, \ac{OSM} y OpenStack.
	\item Desarrollo del cliente para interactuar con \ac{ONOS} (J-ONOS Client).
	\item Desarrollo del cliente para interactuar con \ac{OSM} (J-OSM Client).
	\item Desarrollo del cliente para interactuar con OpenStack (J-OpenStack Client);
	\item Desarrollo del plugin de Net2Plan integrando los clientes desarrollados.
	\item Realización la prueba de concepto y obtención de resultados.
	\item Análisis de resultados y obtención de conclusiones.
	\item Escritura de la memoria.
\end{itemize}

\clearpage

\section{Estructura de la memoria}

El resto de la memoria de este Trabajo de Fin de Máster se ha estructurado de la siguiente manera:
\begin{itemize}

	
	\item Capítulo 2. Estado del arte
	
	En este capítulo se lleva a cabo una explicación de como herramientas open-source han cambiado el concepto de red de telecomunicación. Así mismo, se realiza una extensa explicación de los paradigmas \ac{SDN} y \ac{NFV}.
	
	\item Capítulo 3. Herramientas utilizadas
	
	En este capítulo se realiza una explicación de todas y cada una de las herramientas y librerías que se han utilizado para llevar a cabo este proyecto.
	
	\item Capítulo 4. Desarrollo de APIs
	
	En este capítulo se realiza una explicación de las \acp{API} desarrolladas para este proyecto (J-OSM Client, J-ONOS Client y J-OpenStack Client), así como del plugin de Net2Plan diseñado para integrar las \acp{API} mencionadas.
	
	\item Capítulo 5. Prueba de concepto
	
	En este capítulo se lleva a cabo una explicación de las diferentes fases de la prueba de concepto, así como de sus resultados finales.
	
	\item Capítulo 6. Conclusiones
	
	En este capítulo se realiza un análisis de los resultados obtenidos. También se establecen futuras líneas de investigación y desarrollo para dar continuidad al proyecto.
	
	
\end{itemize}
\cleardoublepage
	
 % Terminado
\chapter{Estado del arte}
En este capítulo se hablará sobre el contexto en el que se enmarca este proyecto.

En primer lugar, se hará una breve explicación de como las herramientas open-source ayudan a agilizar el desarrollo y la funcionalidad en una red de telecomunicaciones.

Por último, se hace especial mención a los dos paradigmas que motivan este proyecto: SDN y NFV.

\section{Herramientas Open-Source en redes de telecomunicación}

Una red de telecomunicación es un conjunto de medios, tecnologías y protocolos que tienen como finalidad el intercambio de información entre diferentes usuarios.


\section{SDN}
\label{sec:sdn}

\begin{figure}[!ht]
	\centering
	\includegraphics[width=0.75\linewidth]{imagenes/arquitectura_sdn}
	\caption{Arquitectura SDN}
	\label{fig:arquitecturasdn}
\end{figure}


\subsection{OpenFlow}
\label{subsec:openflow}

\section{NFV}
\label{sec:nfv}


\cleardoublepage % Terminado (Pulir referencias de fotos)
\chapter{Herramientas utilizadas}

En este capítulo se va a hacer una descripción de cada una de las herramientas utilizadas en el proyecto, haciendo especial mención a Net2Plan, que es una herramienta cuya funcionalidad es la de emular y planificar una red de telecomunicación. Dicha herramienta constituye la base para el desarrollo de este proyecto, más concretamente de la prueba de concepto.

Además de Net2Plan, se han utilizado otras herramientas para complementar la funcionalidad de Net2Plan y proveerle diferentes funcionalidad enmarcadas en las tecnologías SDN y NFV. Dichas herramientas son Mininet, para emular redes de telecomunicación; ONOS, un controlador SDN; ETSI-OSM, un gestor y orquestador NFV; y OpenStack, una herramienta para emular una infraestructura IT.

Todas estas herramientas operan en sintonía para llevar a cabo una prueba de concepto exitosa.


\section{Net2Plan}
\label{sec:net2plan}

Net2Plan\cite{net2planbib} es una herramienta \textit{open-source} programada en Java dedicada a la planificación, optimización y simulación de redes de comunicaciones desarrollada por el grupo de investigación GIRTEL de la Universidad Politécnica de Cartagena. En sus inicios, fue concebida como una herramienta para docencia sobre redes de comunicaciones. Sin embargo, actualmente se ha convertido en una poderosa herramienta de optimización y planificación de redes, con un repositorio de recursos para la planificación de redes, tanto para el entorno académico como para el entorno de la industria y la empresa.

Net2Plan está basado en una representación de redes con componentes abstractos, tales como nodos, enlaces, demandas o rutas, entre otros. Esto está pensado para poder planificar cualquier tipo de red, sin importar la tecnología que utilice. Para poder personalizar las redes a gusto del usuario, cada componentes permite añadir atributos. Además, hay clases que permiten modelar una tecnología en concreto (redes IP, WDM o escenarios de NFV).

Una ventaja de esta herramienta es que tiene dos modos de uso: mediante interfaz gráfica (GUI) y línea de comandos (CLI). La interfaz gráfica está pensada para utilizar en sesiones de laboratorio como un recurso formativo, o para poder ver más detalladamente la red sobre la que se está trabajando. Por otro lado, el modo línea de comandos facilita los estudios de investigación, ya que permite automatizar ejecuciones de algoritmos o simulaciones mediante scripts. Como se ha hablado antes, ambos modos permiten utilizar Net2Plan en el entorno académico (investigación o enseñanza) y en el entorno de la industria y la empresa.

\begin{figure}[ht!]
	\centering
	\includegraphics[width=1\linewidth]{imagenes/n2p_redes}
	\caption{Ventana \textit{Offline network desing and online network simulation}}
	\label{fig:n2p_redes}
\end{figure}

En la figura \ref{fig:n2p_redes} se puede ver el aspecto de la interfaz gráfica de Net2Plan, donde se muestra una topología de España con sus respectivas tablas que aportan información detallada de cada uno de los componentes.

\section{Mininet}
\label{sec:mininet}

Mininet\cite{mininetbib} es una herramienta programada en Python cuyo objetivo es el de emular redes de telecomunicación. Permite crear redes con \textit{hosts}, \textit{switches}, controladores y enlaces a un alto nivel. Los \textit{hosts} de Mininet corren bajo un sistema operativo Linux, mientras que los \textit{switches} soportan el protocolo OpenFlow (ver \ref{subsec:openflow}) para mayor flexibilidad respecto a la configuración del \textit{routing} y para integrarlos dentro de un escenario SDN (ver \ref{sec:sdn}).

Mininet tiene una gran polivalencia, y eso permite que sea utilizado en diferentes tareas, tales como investigación, desarrollo, aprendizaje o testeo. Gracias a ello, se puede conseguir emular una red con un comportamiento similar a una real.

Sus principales características son:
\begin{itemize}
	\item Provee un amplio banco de pruebas para desarrollar aplicaciones basadas en OpenFlow.
	\item Permite que varios desarrolladores trabajen de forma concurrente sobre la misma topología de red.
	\item Permite realizar tests exhaustivos de topologías sin necesidad de tener una real.
	\item Incluye una Interfaz de Línea de Comandos que es independiente de la topología emulada y del protocolo que ésta utilice.
	\item Permite crear desde topologías mas sencillas con un único comando hasta topologías realmente complejas haciendo uso de una API programada en Python para definir los componentes con total detalle.
\end{itemize}

Las redes emuladas por Mininet ejecutan aplicaciones estandarizadas de Linux, como el kernel del propio sistema Linux. Esto permite que cualquier desarrollo llevado a cabo y testeado en Mininet pueda ser movido a un sistema real realizando las mínimas modificaciones posibles.

\section{ONOS}
\label{sec:onos}

ONOS (\textit{Open Network Operative System})\cite{onosbib} es un proyecto Open-Source perteneciente a The Linux Foundation. Su principal objetivo es el de crear un controlador SDN para proveedores de servicios de comunicaciones.

Sus principales características son:
\begin{itemize}
	\item \textbf{Escalabilidad:} Ofrece replicación ilimitada mediante virtualización para poder añadir y quitar capacidad al plano de control según sea necesario.
	\item \textbf{Resiliencia:} Provee la disponibilidad requerida por los operadores de red en momentos críticos.
	\item \textbf{Retrocompatibilidad:} Permite añadir o configurar dispositivos y servicios con configuración basada en modelos.
	\item \textbf{Soporte a dispositivos de nueva generación:} Ofrece control en \textit{real-time} para dispositivos OpenFlow.
	\item \textbf{Modularidad:} Las funcionalidades de ONOS están definidas en modulos localizados, lo que hace más fácil probar y mantener el software en buen estado.
\end{itemize}


ONOS está programado en Java y opera como un clúster de nodos idénticos en cuanto al software. Trabaja con modelos y protocolos estandarizados, tales como OpenFlow (ver \ref{subsec:openflow}), NETCONF, OpenConfig, OpenROADM, entre otros.

\clearpage


\begin{figure}[!ht]
	\centering
	\includegraphics[width=0.7\linewidth]{imagenes/onos_architecture}
	\caption{Arquitectura de ONOS. 
		Fuente: \cite{onostutbib}}
	\label{fig:onosarch}
\end{figure}

En la figura \ref{fig:onosarch} se observa la arquitectura interna de ONOS. En el \textit{Core} se encuentran los controladores de los diferentes servicios que ofrece (TopologyService, DeviceService, HostService, ...), cada uno de ellos destinado a controlar un tipo de componente.

También se puede observar que, para acceder a estos controladores, las aplicaciones necesitan hacer uso de la interfaz \textit{NorthBound}, que se compone de varias subinterfaces que exportan diferentes APIs, como una API REST o una TAPI (\textit{Transport API}).

Por otro lado, para que los controladores puedan tener constancia de los dispositivos de la red, la interfaz \textit{SouthBound} incluye diferentes \textit{drivers}, genéricos o particulares, para poder comunicarse con dispositivos mediante numerosos protocolos estandarizados, como se explicó anteriormente.

\begin{figure}[!ht]
	\centering
	\includegraphics[width=0.8\linewidth]{imagenes/onos_gui}
	\caption{Interfaz Gráfica de ONOS. 
		Fuente: \cite{wikionosbib}}
	\label{fig:onosgui}
\end{figure}

Para facilitar la interacción con el usuario, ONOS ofrece una GUI, como se puede ver en la figura \ref{fig:onosgui}, para ver en más detalle la topología que esta siendo gestionada, así como datos más específicos de cada uno de los dispositivos de la red. 


\subsection{OpenAPI y Swagger}
\label{subsec:openapi}

OpenAPI\cite{openapibib} es una iniciativa creada por varios expertos de la industria y la investigación para estandarizar las descripciones de las RestAPIs. Su principal objetivo es crear y promover un formato de descripción genérico.

Swagger\cite{swaggerbib} es un conjunto de herramientas open-source para definir y documentar REST APIs.

Mediante la colaboración entre OpenAPI, que establece un modelo de APIs común, y Swagger, que permite diseñar una REST API de forma simple, las aplicaciones podrán conectarse entre sí de forma sencilla, y ayudará a tener un mundo más comunicado.

\section{OSM}
\label{sec:osm}

OSM (\textit{Open Source MANO})\cite{osmbib} es un software \textit{open-source} cuya función principal es la orquestación de servicios de red avanzados en infraestructuras NFV heterogéneas. Surge como iniciativa de la ETSI para crear una arquitectura NFV común para los operadores de red.

OSM trabaja con una serie de componentes/elementos que ayudan a definir su arquitectura:

\begin{itemize}
	\item \textbf{VDU (Virtual Deployment Unit):} Es el componente más básico de la arquitectura OSM. Se encarga de definir una máquina virtual. 
	
	\item \textbf{VLD (Virtual Link Descriptor):} Define las conexiones directas entre diferentes componentes de la arquitectura. Hay principalmente dos tipos de VLD: VDU-VDU y VNF-VNF.
	
	\item \textbf{VNFD (Virtual Network Function Descriptor):} Se encarga de definir los recursos necesarios para instanciar un VNF. Incluye diferentes componentes: lista de VDUs que lo definen, lista de VLDs, entre otros.
	
	\item \textbf{NSD (Network Service Descriptor):} Este componente se encarga de definir la configuración de un NS. Incluye diferentes componentes: lista de VNFDs que lo componen, lista de VLDs, parámetros de configuración iniciales, entre otros.
	
	\item \textbf{VNF (Virtual Network Function):} Es el componente que define una función de red virtualizada. Puede estar compuesto de un único VDU o por más de uno.
	
	\item \textbf{NS (Network Service):} Se compone de uno o más VNFs que realizan una función de red más avanzada conjuntamente.
	
	\item \textbf{VIM (Virtual Infrastructure Manager):} Este elemento define un controlador de una o más infraestructuras donde se alojaran las diferentes máquinas virtuales. Es el encargado de comunicar a OSM con las diferentes infraestructuras NFV.
\end{itemize}

\clearpage

\begin{figure}[!ht]
	\centering
	\includegraphics[width=0.8\linewidth]{imagenes/osm_arch}
	\caption{Arquitectura de OSM. 
		Fuente: \cite{osmwikibib}}
	\label{fig:osmarch}
\end{figure}

Para ayudar a explicar el funcionamiento de OSM, la figura \ref{fig:osmarch} da una visión general de la interaccion entre OSM y los diferentes componentes:

\begin{itemize}
	\item \textbf{Interfaz \textit{NorthBound}:} OSM exporta una RestAPI gracias a su interfaz \textit{NorthBound}. Mediante llamadas HTTP (GET, POST, DELETE), el usuario es capaz de ejecutar órdenes en OSM, tales como crear un nuevo VIM o instanciar un nuevo VNF, entre otras.
	
	Para ello, es necesario tener un cliente desde el cuál enviar órdenes. La ETSI ofrece una interfaz gráfica web que se instala al mismo tiempo que OSM y un cliente por línea de comando escrito en Python (ver \ref{subsec:osmclientpython}).
	
	\item \textbf{Conexión con VIM:} OSM permite la comunicación con múltiples tipos de VIM (OpenStack, OpenVIM, VMWare y Amazon Web Services). Para ello, es necesaria conectividad IP entre OSM y el propio VIM, ya que las órdenes enviadas por OSM al VIM para realizar operaciones son hechas mediante una RestAPI.
	
	\item \textbf{Conexión VIM-NFVI:} NFVI (NFV Infrastructure) es el conjunto de recursos (RAM, CPU, HD, entre otros) que son utilizados para instanciar las diferentes máquinas virtuales. El VIM actúa como un controlador de las diferentes NFVIs y gestiona las interacciones entre OSM y ellas.
	
	En estructuras de trabajo pequeñas, es habitual que un VIM y su NFVI estén en la misma máquina física, aunque para estructuras reales de trabajo, la NFVI de un VIM puede estar distribuida en diferentes máquinas físicas, e incluso tener más de una infraestructura.
	
	\item \textbf{VNF \textit{Management}:} cuando un VIM instancia un nuevo VNF, se le asigna una dirección IP para poder acceder a la propia máquina virtual y gestionarla. Por ello, es necesario que haya conectividad IP entre OSM y todos los VNFs. 
\end{itemize}

\subsection{OSMClient}
\label{subsec:osmclientpython}

OSMClient\cite{osmclientbib} es una cliente REST programado en Python por la ETSI para interactuar con OSM. Fue introducido en la \textit{release} 2 y permite al usuario ejecutar numerosas acciones, como subir un nuevo descriptor, instanciar un nuevo NS, eliminar un NS, entre otras.

En sus orígenes, cuando la \textit{release} 2 de OSM fue dispuesta al público, dicho cliente tenía una funcionalidad limitada y únicamente podía ser instalado junto a la versión completa de OSM.

Cuando la ETSI publicó la \textit{release} 5 de OSM, el cliente sufrió un gran lavado de cara, al igual que OSM, el cual tuvo un gran cambio interno y externo. Se añadió una nueva versión del cliente que seguía el estándar sol005\cite{sol005item}, el cual define un nuevo formato de REST APIs para aplicaciones propias de la ETSI.


\section{OpenStack}
\label{sec:openstack}

OpenStack\cite{openstackbib} es un proyecto que sigue el paradigma \textbf{Cloud Computing} para proporcionar un \textit{datacenter} para controlar y gestionar grandes cantidades de recursos de computación, almacenamiento y red. Para facilitar la gestión de recursos, OpenStack provee al usuario una interfaz gráfica a la vez que también exporta una RestAPI para permitir conectividad con aplicaciones externas.

\begin{figure}[!ht]
	\centering
	\includegraphics[width=0.9\linewidth]{imagenes/openstack_arch}
	\caption{Arquitectura de OpenStack. Fuente:\cite{openstackbib}}
	\label{fig:openstackarch}
\end{figure}

Está compuesto de diferentes servicios o bloques, encargándose cada uno de ellos de una funcionalidad concreta dentro de la arquitectura. La principal característica de estos servicios es que son accesibles de forma independiente. Los servicios principales de OpenStack son los siguientes:

\begin{itemize}
	\item \textbf{Keystone:}  Este servicio controla la identificación de los diferentes usuarios que se conecten a la infraestructura de OpenStack, y el acceso a según que aplicaciones de los mismos.
	
	\item \textbf{Horizon:} Este servicio es el encargado de mostrar la gestión completa de OpenStack mediante una interfaz gráfica. Desde ella se puede observar con todo detalle que está sucediendo en el sistema y poder gestionar los posibles fallos.
	
	\item \textbf{Nova:} Este servicio está considerado el motor de OpenStack. Es el encargado de desplegar y administrar las diferentes máquinas virtuales instanciadas y otros servicios que se necesiten.
	
	\item \textbf{Neutron:} Este servicio es el encargado de que cada componente desplegado en OpenStack se comunique con los demas y estén interrelacionados.
	
	\item \textbf{Glance:} Este servicio se encarga de gestionar las diferentes imágenes que se usan en la infraestructura.
	
	\item \textbf{Cinder:} Este servicio se centra en el almacenamiento. Facilita el acceso al contenido alojado en las unidades de disco que se encuentren en la infraestructura.
	
	\item \textbf{Swift:} Este servicio es el encargado de almacenar los diferentes archivos del sistema, asegurar su integridad y replicarlos por los diferentes discos de la infraestructura, para hacer más dinámicas la accesibilidad y la disponibilidad.
\end{itemize}

\subsection{OpenStack4Java}
\label{subsec:openstack4j}

OpenStack4Java\cite{openstack4jbib} es una librería REST \textit{open-source} programada en Java para controlar y gestionar un sistema basado en OpenStack.

Permite al usuario realizar una gestión de OpenStack eficiente gracias a sus múltiples módulos, cada uno de ellos focalizado en gestionar un servicio concreto de OpenStack:

\begin{itemize}
	\item \textbf{Identity:} Este módulo se encarga de gestionar el servicio \textbf{Keystone}. Su principal objetivo es el de gestionar el directorio de usuarios, grupos, regiones, servicios y \textbf{endpoints}. Así mismo, se encarga de autenticar y autorizar a los diferentes usuarios para utilizar los diferentes servicios.
	
	\item \textbf{Compute:} Este módulo se encarga de gestionar el servicio \textbf{Nova}. Es el encargado de gestionar las diferentes máquinas virtuales que están corriendo en OpenStack.
	
	\item \textbf{Network:} Este módulo se encarga de gestionar el servicio \textbf{Neutron}. Provee conectividad entre diferentes componentes de OpenStack. Permite a los usuarios crear sus propias redes y añadirles interfaces.
	
	\item \textbf{Image:} Este módulo se encarga de gestionar el servicio \textbf{Glance}. Su principal funcionalidad es la de proveer diferentes servicios para la gestión de imágenes. Permite almacenar imágenes personalizadas por el usuario para inicializar máquinas rápidamente.
	
	\item \textbf{Block Storage:} Este módulo se encarga de gestionar el servicio \textbf{Cinder}. Permite al usuario crear y montar volúmenes para escalar el almacenamiento.
	
	\item \textbf{Object Storage:} Este módulo se encarga de gestionar el servicios \textbf{Swift}. Es el encargado de crear almacenamiento persistente para los diferentes archivos alojados en el sistema.
\end{itemize}


\begin{figure}[!ht]
	\centering
	\includegraphics[width=1\linewidth]{imagenes/ejemplo_os4j}
	\caption{Ejemplo de uso de OpenStack4j. Fuente:\cite{openstack4jbib}}
	\label{fig:ejemploos4j}
\end{figure}

En la figura \ref{fig:ejemploos4j} se puede ver un breve ejemplo de como utilizar los servicios Identity, Compute, Image y Network.




\cleardoublepage
 % Terminado (Pulir referencias de fotos)
\chapter{Desarrollo de APIs}

\section{ONOS Client}
\label{sec:onosclient}

\section{J-OSM Client}
\label{sec:osmclient}

\section{OpenStack Client}
\label{sec:openstackclient}

\section{Net2Plan: SDN/NFV Management Plugin}
\label{sec:nfvplugin}

En la figura \ref{fig:sdnnfvplugin} se puede observar la vista que ofrece el plugin de Net2Plan desarrollado para la prueba de concepto.

\cleardoublepage % Falta explicar cada api
\chapter{Prueba de concepto}

En este capítulo se va a hablar de una pequeña prueba de concepto para ver el funcionamiento de todas las APIs y herramientas conjuntamente.

Inicialmente, se establece una arquitectura de desarrollo, en la que se define el papel de cada API y herramienta.

Por último, se realiza una explicación del funcionamiento de la prueba de concepto, así como una vista de los resultados finales.


\section{Arquitectura}
\label{sec:arquitectura}

La arquitectura de la demostración muestra los diferentes elementos que la componen, así como las interacciones entre ellos para el intercambio de datos.

A continuación vemos una explicación de cada elemento perteneciente a la demo y que componente lleva a cabo esa acción:

\begin{itemize}
	\item \textbf{Operation Support System (OSS):} representa el papel de un operador que despliega un servicio gracias a una aplicación. El operador es emulado mediante Net2Plan, más concretamente por su plugin de \textit{NFV Management} (ver sección \ref{sec:nfvplugin}).
	
	\item \textbf{NFV Orchestrator (NFV-O):} representa el papel de una aplicación que se encarga de gestionar la infraestructura de virtualización necesaria para instanciar diferentes máquinas virtuales. ETSI-OSM (ver \ref{sec:osm}) es el encargado de dicha función.
	
	\item \textbf{Virtual Infrastructure Managers (VIMs):} son los encargados de instanciar y alojar las diferentes máquinas virtuales pertenecientes a los VNF. OpenStack (ver \ref{sec:openstack}) es quien realiza este papel.
	
	\item \textbf{Red de Transporte:} la red de transporte es emulada mediante Mininet (ver \ref{sec:mininet}) para establecer flujos de paquetes entre las diferentes VNFs de una Service Chain.
	
	\item \textbf{Controlador SDN:} la red de transporte es controlada por una instancia de ONOS (ver \ref{sec:onos}) mediante el envio de paquetes Openflow (ver \ref{subsec:openflow}) a los diferentes switches de la red.
	
	\item \textbf{Latency-Aware Service Chain Computation Element (LA-SCCE):} Se encarga de decidir el camino a seguir para atravesar una secuencia de VNFs que cumpla con los requisitos de latencia máxima. Este papel lo representa la extensión de Net2Plan mediante la ejecución de un algoritmo de \textit{NFV Placement}. 
\end{itemize}

En la figura \ref{fig:esquemademo} se puede observar un esquema de la arquitectura, en el que se incluyen todos los elementos explicados anteriormente, y hace más fácil de entender como interactúan entre sí los componentes:

\begin{figure}[!ht]
	\centering
	\includegraphics[width=0.7\linewidth]{imagenes/esquema_demo}
	\caption{Arquitectura de la Prueba de Concepto}
	\label{fig:esquemademo}
\end{figure}



\section{Desarrollo}
\begin{itemize}
	\item Paso 1. Haciendo click en el botón LOAD, Net2Plan recibe la información referente a la red de transporte de ONOS, la información sobre los posibles VNFs a instanciar de ETSI-OSM y la información sobre cada VIM de OpenStack, todo mediante llamadas a sus respectivas Rest-APIs.
	
	\item Paso 2. El usuario define la Service Chain que quiere satisfacer (nodo origen, nodo destino, secuencia ordenada de VNFs a atravesar, latecia máxima y ancho de banda) a través de la interfaz gráfica del Plugin.
	
	\item Paso 3. Net2Plan recibe la información introducida por el usuario y la transfiere al LA-SCCE para que ejecute el algoritmo que devolverá como resultado una ruta de enlaces para la Service Chain y una serie de VNFs instanciadas en diferentes VIMs.
	
	\item Paso 4. Net2Plan envía la orden a ETSI-OSM de instanciar las VNFs en los VIMs que el LA-SCCE obtuvo como óptimos.
	
	\item Paso 5. ETSI-OSM envia órdenes a los diferentes VIMs (OpenStack) para que alojen las diferentes máquinas virtuales correspondientes a los VNFs.
	
	\item Paso 6. Net2Plan envía la orden a ONOS con diferentes reglas de flujo para establecer en los diferentes switches de la red de transporte, todo según lo obtenido por el LA-SCCE.
	
	\item Paso 7. Una vez establecidos las reglas de flujo mediante OpenFlow, se realiza una prueba de conexión para asegurar que la Service Chain está establecida.
\end{itemize}

\begin{figure}[!ht]
	\centering
	\includegraphics[width=0.7\linewidth]{imagenes/nfv_service_chain}
	\caption{Estado final de la prueba de concepto}
	\label{fig:nfvservicechain}
\end{figure} % Terminado (A falta de ver más fotos para completar la explicación)
\chapter{Conclusiones}
\label{conclusiones}

El objetivo inicialmente propuesto fue el desarrollo de diferentes \acp{API} para comunicar diferentes herramientas \textit{open-source} entre sí para poder diseñar un escenario \ac{SDN}-\ac{NFV} heterogéneo.

Una vez establecidos los objetivos, se comenzó con el desarrollo de J-OSMClient, J-ONOSClient y J-OpenStackClient para poder establecer la comunicación con \ac{OSM}, \ac{ONOS} y OpenStack, gracias a las \acp{API} \ac{REST} que exportan cada uno de ellos.

Una vez los clientes estuvieron totalmente funcionales, se comenzó con el desarrollo del plugin \textit{NFV Management}, para convertir a Net2Plan en la entidad central que permitierá comunicar a las diferentes herramientas entre sí y hacerlas trabajar en sintonía.

Cuando el plugin estuvo totalmente desarrollado, se diseño una prueba de concepto para ser presentada en el congreso \ac{ECOC} en Septiembre de 2018 para demostrar como las diferentes herramientas trabajaban conjuntamente para satisfacer diferentes \textit{Service Chains} con requisitos de latencia y ancho de banda, anticipándo la llegada del 5G.


\section{Fortalezas}

Inicialmente, se pretendía desarrollar un conjunto de \acp{API} para poder diseñar un entorno heterogéneo donde las tecnologías \ac{SDN} y \ac{NFV} cumplieran un papel fundamental en él, mediante diferentes herramientas que siguen dichas tecnologías operando en total sintonía.

Una vez acabado el proyecto, cabe decir que la principal fortaleza de este Trabajo de Fin de Máster es que los diferentes objetivos propuestos al inicio se han conseguido satisfactoriamente, cumpliendo las expectativas puestas al comiendo del mismo.

\section{Análisis de resultados}

Una vez acabado el proyecto y habiendo obtenido resultados para su correspondiente análisis, se puede observar como, gracias a J-ONOSClient y a J-OpenStackClient, el plugin de Net2Plan puede obtener información exhaustiva sobre la red de transporte controlada por \ac{ONOS} y de los recursos internos de los \acp{VIM}, permitiendo utilizar dicha información como parámetros de entrada del algoritmo de planificación ejecutado por el \ac{LA-SCCE}, lo que conlleva unos resultados más realistas.

En referencia a J-OSMClient, hay que mencionar que es el único cliente \textit{open-source} programado en Java que existe para establecer comunicación con \ac{OSM}. Aunque existe el cliente en Python desarrollado por la \ac{ETSI}, dicho cliente no permite utilizarse de manera gráfica, debido a su naturaleza de \ac{CLI}. 

Por ello, la creación de J-OSMClient proporciona una amplio abanico de trabajo, permitiendo que \ac{OSM} sea gestionado por una aplicación externa.

Para finalizar el análisis de resultados, hay que mencionar los resultados obtenidos al realizar la prueba de concepto. Una vez instanciados los \acp{VNF} correspondientes a la \textit{Service Chain} y habiendo instalado las reglas de flujo correspondientes a la ruta óptima obtenida en los \textit{switches}, se realizan una prueba de conexión enviando un ping entre el origen y el destino. 

En la figuras \ref{fig:nfvservicechain} y \ref{fig:topo_onos} se puede observar como la ruta es la misma en ambos casos (en el plugin de Net2Plan y en \ac{ONOS}), y en la figura \ref{fig:onosflowrules} como las reglas de flujo aplicadas indican que han procesado un paquete, que corresponde al ping realizado anteriormente para validad la conectividad.


\section{Líneas de trabajo futuro y mejoras}

Aunque el objetivo de este proyecto se ha cumplido con creces, siempre se puede mejorar. Por ello, se proponen las siguientes líneas de trabajo futuro y mejoras:

\begin{itemize}
	
	\item Emular una red de transporte multicapa (\ac{IP} sobre \ac{WDM}) para conseguir un escenario más realista. Para ello, evaluar herramientas como LINC-OE\cite{lincoebib} o incluso utilizar agentes basados en modelos \ac{YANG}.
	
	\item Complementar la prueba de concepto con herramientas que sirvan para monitorizar el estado interno de los \acp{VIM} para obtener información interna con más nivel de detalle.
	
	\item Actualmente, J-ONOSClient y J-OpenStackClient se encuentran dentro del código del plugin \textit{NFV Management}, y este se encuentra bajo un repositorio Git privado. Sería útil exportar ambos clientes a GitHub para que estuvieran disponibles para utilizar en cualquier otra herramienta para otros escenarios.
	
\end{itemize}

\cleardoublepage % Sin empezar
%----------------------------------------------------------------------------------------
%	THESIS CONTENT - APPENDICES
%----------------------------------------------------------------------------------------

\appendix % Cue to tell LaTeX that the following "chapters" are Appendices
\clearpage
\chapter{Script Mininet Red Transporte}
\label{sec:scriptmininet}

\begin{lstlisting}
from mininet.net import Mininet
from mininet.node import Controller, RemoteController, OVSController
from mininet.link import Intf
from mininet.cli import CLI
from mininet.nodelib import NAT
import time

class RedEspana():

def __init__(self):

net = Mininet()

controller_onos = net.addController('controller', 
controller = RemoteController, ip = '10.0.2.11')

numberElements = 21

switches = []
hosts = []
vimIndexes = [8,19]

for s in range(numberElements):
	switch = net.addSwitch('s'+str(s+1))
	switches.append(switch)


for h in range(numberElements):
	if h not in vimIndexes:
		host = net.addHost('h'+str(h+1), 
		ip = '15.0.0.'+str(h+1+50)+'/24')
		hosts.append(host)
		net.addLink(host, switches[h])


net.addLink(switches[0], switches[1])
net.addLink(switches[0], switches[2])
net.addLink(switches[1], switches[2])
net.addLink(switches[1], switches[3])
net.addLink(switches[2], switches[4])
net.addLink(switches[2], switches[6])
net.addLink(switches[3], switches[4])
net.addLink(switches[3], switches[9])
net.addLink(switches[4], switches[5])
net.addLink(switches[4], switches[7])
net.addLink(switches[5], switches[8])
net.addLink(switches[6], switches[8])
net.addLink(switches[6], switches[14])
net.addLink(switches[7], switches[8])
net.addLink(switches[7], switches[10])
net.addLink(switches[7], switches[11])
net.addLink(switches[8], switches[12])
net.addLink(switches[9], switches[10])
net.addLink(switches[9], switches[20])
net.addLink(switches[10], switches[19])
net.addLink(switches[10], switches[20])
net.addLink(switches[11], switches[12])
net.addLink(switches[11], switches[18])
net.addLink(switches[11], switches[19])
net.addLink(switches[12], switches[13])
net.addLink(switches[12], switches[17])
net.addLink(switches[13], switches[14])
net.addLink(switches[13], switches[16])
net.addLink(switches[14], switches[15])
net.addLink(switches[15], switches[16])
net.addLink(switches[16], switches[17])
net.addLink(switches[17], switches[18])
net.addLink(switches[18], switches[19])
net.addLink(switches[19], switches[20])

intf_vimone = Intf('enx0050b6253bb0',switches[8])
intf_vimtwo = Intf('enx0050b6253baf',switches[19])

print('Added physical interfaces '+str(intf_vimone)+' and '+str(intf_vimtwo))

controller.start()

net.start()

time.sleep(3)

net.pingAll()

for host in hosts:
	host.cmd('ip route add 15.0.2.0/24 via 15.0.0.59')
	host.cmd('ip route add 15.0.3.0/24 via 15.0.0.70')
	host.cmd('/usr/sbin/sshd -D&')
	ping_vimone = host.cmd('ping 15.0.0.59 -c 1')
	print(str(ping_vimone))
	ping_vimtwo = host.cmd('ping 15.0.0.70 -c 1')
	print(str(ping_vimtwo))

CLI(net)




topos = { 'mytopo' : ( lambda : RedEspana() ) }
\end{lstlisting}


\cleardoublepage

\clearpage



%----------------------------------------------------------------------------------------
%	BIBLIOGRAPHY
%----------------------------------------------------------------------------------------

%\printbibliography[heading=bibintoc]
%\addcontentsline{toc}{chapter}{Bibliografía}
%\bibliographystyle{IEEEtran}
%\bibliography{referencias}

%----------------------------------------------------------------------------------------


\end{document}  
