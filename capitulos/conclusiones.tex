\chapter{Conclusiones}
\label{conclusiones}

El objetivo inicialmente propuesto fue el desarrollo de diferentes \acp{API} para comunicar diferentes herramientas \textit{open-source} entre sí para poder diseñar un escenario \ac{SDN}-\ac{NFV} heterogéneo.

Una vez establecidos los objetivos, se comenzó con el desarrollo de J-OSMClient, J-ONOSClient y J-OpenStackClient para poder establecer la comunicación con \ac{OSM}, \ac{ONOS} y OpenStack, gracias a las \acp{API} \ac{REST} que exportan cada uno de ellos.

Una vez los clientes estuvieron totalmente funcionales, se comenzó con el desarrollo del plugin \textit{NFV Management}, para convertir a Net2Plan en la entidad central que permitierá comunicar a las diferentes herramientas entre sí y hacerlas trabajar en sintonía.

Cuando el plugin estuvo totalmente desarrollado, se diseño una prueba de concepto para ser presentada en el congreso \ac{ECOC} en Septiembre de 2018 para demostrar como las diferentes herramientas trabajaban conjuntamente para satisfacer diferentes \textit{Service Chains} con requisitos de latencia y ancho de banda, anticipándo la llegada del 5G.


\section{Fortalezas}

Inicialmente, se pretendía desarrollar un conjunto de \acp{API} para poder diseñar un entorno heterogéneo donde las tecnologías \ac{SDN} y \ac{NFV} cumplieran un papel fundamental en él, mediante diferentes herramientas que siguen dichas tecnologías operando en total sintonía.

Una vez acabado el proyecto, cabe decir que la principal fortaleza de este Trabajo de Fin de Máster es que los diferentes objetivos propuestos al inicio se han conseguido satisfactoriamente, cumpliendo las expectativas puestas al comiendo del mismo.

\section{Análisis de resultados}

Una vez acabado el proyecto y habiendo obtenido resultados para su correspondiente análisis, se puede observar como, gracias a J-ONOSClient y a J-OpenStackClient, el plugin de Net2Plan puede obtener información exhaustiva sobre la red de transporte controlada por \ac{ONOS} y de los recursos internos de los \acp{VIM}, permitiendo utilizar dicha información como parámetros de entrada del algoritmo de planificación ejecutado por el \ac{LA-SCCE}, lo que conlleva unos resultados más realistas.

En referencia a J-OSMClient, hay que mencionar que es el único cliente \textit{open-source} programado en Java que existe para establecer comunicación con \ac{OSM}. Aunque existe el cliente en Python desarrollado por la \ac{ETSI}, dicho cliente no permite utilizarse de manera gráfica, debido a su naturaleza de \ac{CLI}. 

Por ello, la creación de J-OSMClient proporciona una amplio abanico de trabajo, permitiendo que \ac{OSM} sea gestionado por una aplicación externa.

Para finalizar el análisis de resultados, hay que mencionar los resultados obtenidos al realizar la prueba de concepto. Una vez instanciados los \acp{VNF} correspondientes a la \textit{Service Chain} y habiendo instalado las reglas de flujo correspondientes a la ruta óptima obtenida en los \textit{switches}, se realizan una prueba de conexión enviando un ping entre el origen y el destino. 

En la figuras \ref{fig:nfvservicechain} y \ref{fig:topo_onos} se puede observar como la ruta es la misma en ambos casos (en el plugin de Net2Plan y en \ac{ONOS}), y en la figura \ref{fig:onosflowrules} como las reglas de flujo aplicadas indican que han procesado un paquete, que corresponde al ping realizado anteriormente para validad la conectividad.


\section{Líneas de trabajo futuro y mejoras}

Aunque el objetivo de este proyecto se ha cumplido con creces, siempre se puede mejorar. Por ello, se proponen las siguientes líneas de trabajo futuro y mejoras:

\begin{itemize}
	
	\item Emular una red de transporte multicapa (\ac{IP} sobre \ac{WDM}) para conseguir un escenario más realista. Para ello, evaluar herramientas como LINC-OE\cite{lincoebib} o incluso utilizar agentes basados en modelos \ac{YANG}.
	
	\item Complementar la prueba de concepto con herramientas que sirvan para monitorizar el estado interno de los \acp{VIM} para obtener información interna con más nivel de detalle.
	
	\item Actualmente, J-ONOSClient y J-OpenStackClient se encuentran dentro del código del plugin \textit{NFV Management}, y este se encuentra bajo un repositorio Git privado. Sería útil exportar ambos clientes a GitHub para que estuvieran disponibles para utilizar en cualquier otra herramienta para otros escenarios.
	
\end{itemize}

\cleardoublepage