\chapter{Conclusiones}
\label{conclusiones}

El objetivo inicialmente propuesto fue el desarrollo de diferentes \acp{API} para comunicar diferentes herramientas \textit{open-source} entre sí para poder diseñar un escenario \ac{SDN}-\ac{NFV} heterogéneo.

Una vez establecidos los objetivos, se comenzó con el desarrollo de J-OSMClient, J-ONOSClient y J-OpenStackClient para poder establecer la comunicación con \ac{OSM}, \ac{ONOS} y OpenStack, gracias a las \acp{API} \ac{REST} que exportan cada uno de ellos.

Una vez los clientes estuvieron totalmente funcionales, se comenzó con el desarrollo del plugin \textit{NFV Management}, para convertir a Net2Plan en la entidad central que permitierá comunicar a las diferentes herramientas entre sí y hacerlas trabajar en sintonía.

Cuando el plugin estuvo totalmente desarrollado, se diseño una prueba de concepto para ser presentada en el congreso \ac{ECOC} en Septiembre de 2018 para demostrar como las diferentes herramientas trabajaban conjuntamente para satisfacer diferentes \textit{Service Chains} con requisitos de latencia y ancho de banda, anticipándo la llegada del 5G.


\section{Fortalezas}

Inicialmente, se pretendía desarrollar un conjunto de \acp{API} para poder diseñar un entorno heterogéneo donde las tecnologías \ac{SDN} y \ac{NFV} cumplieran un papel fundamental en él, mediante diferentes herramientas que siguen dichas tecnologías operando en total sintonía.

Una vez acabado el proyecto, cabe decir que la principal fortaleza de este Trabajo de Fin de Máster es que los diferentes objetivos propuestos al inicio se han conseguido satisfactoriamente, cumpliendo las expectativas puestas al comiendo del mismo.

\section{Análisis de resultados}

\section{Trabajo futuro y mejoras}

Como propuestas de mejora y trabajo futuro, se proponen las siguientes:

\begin{itemize}
	
	\item Emular una red de transporte multicapa (\ac{IP} sobre \ac{WDM}) para conseguir un escenario más realista. Para ello, evaluar herramientas como LINC-OE\cite{lincoebib} o incluso utilizar agentes basados en modelos \ac{YANG}.
	
	\item Complementar la prueba de concepto con herramientas que sirvan para monitorizar el estado interno de los \acp{VIM}.
	
\end{itemize}

\cleardoublepage