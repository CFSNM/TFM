\chapter{Introducción}

El paradigma SDN se ha convertido en la manera más eficiente de gestionar las redes actuales, gracias a la automatización de las funciones de operación y gestión. A su vez, el paradigma NFV se ha convertido en un recurso valioso para optimizar recursos de una infraestructura IT. 

Debido a esto, han ido surgiendo herramientas para la aplicación de las técnicas SDN y NFV para automatizar y optimizar las redes de telecomunicación.

\section{Motivaciones}

Este proyecto viene motivado por las diferentes herramientas que existen para aplicar técnicas SDN y NFV a las redes de telecomunicación. Dichas herramientas tienen interfaces para permitir al usuario el poder comunicarse con ellas, pero son muy diferentes entre sí. 

Así mismo, no existe una entidad central que permita gestionar las interacciones entre ellas de forma óptima. Por esto último surge la idea de desarrollar diferentes librerías (APIs) \textbf{open-source} para interactuar con las diferentes herramientas de una forma sencilla y transparente para el usuario, y convertir a Net2Plan en esa entidad central que se pueda comunicar con las diferentes herramientas y gestionar las interacciones entre ellas.

\section{Objetivos}

El objetivo principal de este trabajo es desarrollar diferentes APIs \textbf{open-source} e integrarlas en un plugin de Net2Plan para llevar a cabo una prueba de concepto. Este objetivo se puede desglosar en otros más pequeños:

\begin{itemize}
	\item Adquirir amplio conocimiento de las distintas herramientas (ONOS, Mininet, OSM y OpenStack). 
	\item Desarrollo de APIs para entornos SDN (ONOS Client).
	\item Desarrollo de APIs para entornos NFV (J-OSM Client y OpenStack Client).
	\item Integración de todas las APIs en un plugin de Net2Plan para una prueba de concepto.
	\item Obtención y análisis de resultados.
\end{itemize}

\section{Plan de trabajo}

Para la consecución de los objetivos marcados, el plan de trabajo del proyecto consta de diferentes fases, cada una de ellas destinada a cumplir un objetivo concreto:

\begin{itemize}
	\item Estudio y familiarización con ONOS, Mininet, OSM y OpenStack.
	\item Desarrollo de la librería para interactuar con ONOS (ONOS Client).
	\item Desarrollo de las lubrerías para interactucar con OSM y OpenStack (J-OSM Client y OpenStack Client).
	\item Desarrollo del plugin de Net2Plan integrando las librerías desarrolladas.
	\item Realizar la prueba de concepto y obtener conclusiones.
	\item Escritura de la memoria.
\end{itemize}

\section{Estructura de la memoria}

La memoria de este Trabajo de Fin de Máster se estructura de la siguiente manera:
\begin{itemize}
	
	\item Capítulo 1. Introducción
	
	En este capitulo se realiza una breve contextualización del proyecto, detallando las motivaciones, los objetivos y el plan de trabajo.
	
	\item Capítulo 2. Estado del arte
	
	En este capítulo se lleva a cabo una explicación de como herramientas open-source han cambiado el concepto de red de telecomunicación. Así mismo, se realiza una extensa explicación de los paradigmas SDN y NFV.
	
	\item Capítulo 3. Herramientas utilizadas
	
	En este capítulo se realiza una explicación de todas y cada una de las herramientas y librerías que se han utilizado para llevar a cabo este proyecto.
	
	\item Capítulo 4. Desarrollo de APIs
	
	En este capítulo se realiza una explicación de las APIs desarrolladas para este proyecto (J-OSM Client, ONOS Client y OpenStack Client), así como del plugin de Net2Plan diseñado para integrar las APIs mencionadas.
	
	\item Capítulo 5. Prueba de concepto
	
	En este capítulo se lleva a cabo una explicación de las diferentes fases de la prueba de concepto, así como de sus resultados finales.
	
	\item Capítulo 6. Conclusiones
	
	En este capítulo se realiza un análisis de los resultados obtenidos. También se establecen futuras líneas de investigación y desarrollo para dar continuidad al proyecto.
	
	
\end{itemize}
\cleardoublepage
	
