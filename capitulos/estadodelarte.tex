\chapter{Estado del arte}
En este capítulo se hablará sobre el contexto en el que se enmarca este proyecto.

En primer lugar, se hará una breve explicación de como las herramientas open-source ayudan a agilizar el desarrollo y la funcionalidad en una red de telecomunicaciones.

Por último, se hace especial mención a los dos paradigmas que motivan este proyecto: SDN y NFV.

\section{Herramientas Open-Source en redes de telecomunicación}

Una red de telecomunicación es un conjunto de medios, tecnologías y protocolos que tienen como finalidad el intercambio de información entre diferentes usuarios.

Así mismo, se está produciendo una enorme evolución en el concepto de una red de telecomunicación. 

Antes, este concepto era puramente físico, con un conjunto de dispositivos \textit{Hardware}, como pueden ser \textit{routers}, \textit{switches} u ordenadores, interactuando entre sí. Actualmente, prácticamente el 100\% de las redes utilizan software open-source para diferentes propósitos:

\begin{itemize}
	\item Sacar el máximo rendimiento a su infraestructura.
	\item Agilizar el envío y procesamiento del tráfico de la red.
	\item Acelerar y automatizar la gestión y configuración de los dispositivos.
	\item Reducir los costes de operación de la red.
\end{itemize}

Para conseguir los propósitos mencionados anteriormente, existen numerosas herramientas de software desarrolladas por empresas, universidades u organizaciones que están totalmente disponibles para ser usadas por cualquier usuario.

Dichas herramientas forman un gran conjunto heterogéneo, siendo desarrollada cada una de ellas para uno o más propósitos:

\begin{itemize}
	\item Para sacar el máximo rendimiento a la infraestructura de una red, existen herramientas de virtualización como \textbf{OpenStack} (ver \ref{sec:openstack}) o \textbf{Docker}, para proveer a las aplicaciones de una abstracción e independencia. Este tipo de herramientas pertenecen al paradigma \textbf{NFV} (ver \ref{sec:nfv}).
	
	\item Para agilizar el envío y procesamiento del tráfico de una red, existen herramientas que permiten crear dispositivos hardware virtualizados como \textbf{OpenVSwitch} o \textbf{Mininet} (ver \ref{sec:mininet}). Gracias a este tipo de herramientas, se puede sustituir un \textit{Switch} clásico por un pequeño software que hace las funciones de un \textit{Switch} de una manera más eficiente.
	
	\item La gestión de los dispositivos se ha realizado de forma manual, dispositivo a dispositivo. Gracias a herramientas software como \textbf{Cacti} o \textbf{Nagios}, la forma de gestionar las redes de telecomunicación ha dado un giro de 180 grados. Utilizando estas herramientas, el usuario puede gestionar la red desde un terminal de manera remota, gracias al protocolo SNMP (\textbf{Simple Network Management Protocol}).
	
	\item Para acelerar y automatizar la configuración de los dispositivos de una red de telecomunicación, existen herramientas que pertenecen al paradigma \textit{SDN} (ver \ref{sec:sdn}). Un ejemplo de estas herramientas es el controlador SDN ONOS (ver \ref{sec:onos}).
	
	\item Para reducir los costes de operación de una red, existen herramientas open-source que ayudan a planificar una red de telecomunicación de forma óptima. La más conocida de estas herramientas es \textbf{Net2Plan} (ver \ref{sec:net2plan}), una herramienta de planificación de redes programada en Java.
\end{itemize}

\section{SDN}
\label{sec:sdn}

SDN (\textit{Software Defined Networking}) es un paradigma que consiste en una nueva forma de configurar las redes de telecomunicación. 

Su principal premisa es la de separar el plano de control (\textit{Software}) del plano de datos (\textit{Hardware}).

SDN pretende cambiar la manera tradicional de configuración de dispositivos usando instrucciones de bajo nivel por una manera novedosa mediante herramientas software a un alto nivel.

\subsection{Arquitectura SDN}

Las redes definidas por \textit{software} constan de una arquitectura de red específica, como se puede ver en la figura \ref{fig:arquitecturasdn}.
 
\begin{figure}[!ht]
	\centering
	\includegraphics[width=0.8\linewidth]{imagenes/arquitectura_sdn}
	\caption{Arquitectura SDN}
	\label{fig:arquitecturasdn}
\end{figure}

Los diferentes elementos que componen la arquitectura SDN son los siguientes:

\begin{itemize}
	\item \textbf{Controlador SDN:} Es el cerebro de la red SDN. Forma el núcleo de la arquitectura SDN comunicándose con los \textit{switches} a través de la \textbf{Interfaz Sur} (\textit{SouthBound Interface}) y con las distintas aplicaciones a través de la \textbf{Interfaz Norte} (\textit{NorthBound Interface}).
	
	\item \textbf{Interfaz Sur:} Es la interfaz que conecta al controlador SDN con el plano de datos. Facilita la configuración de la red, transfiriendo dichas configuraciones a los dispositivos de la red. El protocolo más utilizado en esta interfaz es \textbf{OpenFlow} (ver \ref{subsec:openflow}).
	
	\item \textbf{Interfaz Norte:} Es la interfaz que conecta al controlador SDN con el plano de control. Facilita el proceso de automatización de la red mediante la comunicación del controlador con las diferentes aplicaciones. Para permitir dicha comunicación con las aplicaciones, la interfaz norte exporta una REST-API.
	
	\item \textbf{Agentes y \textit{Drivers}:} Para establecer la comunicación entre el controlador y los dispositivos mediante la interfaz sur y para la comunicación entre el controlador y las aplicaciones mediante la interfaz norte,es necesario un par agente-\textit{driver}. El \textit{driver} se encuentra en el controlador, mientras que el agente se encuentra en el dispositivo. Se encargan de realizar la comunicación en el plano de datos, realizando la conversión del lenguaje del controlador al del dispositivo, y viceversa.
	
	\item \textbf{Aplicaciones SDN:} Son programas que se conectan al controlador SDN mediante la interfaz norte. Gracias a su lógica de aplicaciones, desarrollan un próposito concreto y transfieren datos u órdenes al controlador SDN.
\end{itemize}


\subsection{OpenFlow}
\label{subsec:openflow}

\section{NFV}
\label{sec:nfv}


\cleardoublepage